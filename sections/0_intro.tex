\sectioncentered*{Введение}
\addcontentsline{toc}{section}{ВВЕДЕНИЕ}

В настоящее время все более актуальной становится задача защиты информационных ресурсов компьютерных сетей от атак со стороны внутренних или внешних нарушителей. Это связано прежде всего со стремительным развитием информационных технологий, охватывающих многие сферы жизни общества и государства в целом. В повседневной жизни Интернет используется не только для сбора информации, но и для работы с централизованными базами данных, для работы с клиентами, для объединения филиалов и удаленных подразделений в единую сеть, и так далее. Поэтому в настоящее время остро стоят вопросы  эффективной и гибкой защиты ресурсов корпоративной сети организации при ее подключении к глобальной сети от внешних нарушителей и локальной сети при угрозе от внутренних нарушителей \citep{emelyanova_fralenko}.

Злоумышленник всегда первым выбирает цель и методы осуществления атаки, а система защиты не имеет информации о его выборе до тех пор, пока не будет зарегистрировано вторжение. Для решения данной проблемы необходимо не только предупреждать, блокировать, обнаруживать и реагировать на действия нарушителей, но и отвлекать их от основных целей, заманивая на ложные информационные объекты, производить сбор информации о приемах, тактике и мотивации злоумышленников, осуществлять их идентификацию и разоблачение. Именно поэтому все более актуальной задачей становится внедрение обманных систем в защищаемую сеть. Другими словами, такие ложные информационные системы или по-другому — обманные системы («ловушки») создаются для привлечения злоумышленников и удерживают их внимание на «приманке». А пока злоумышленник будет атаковать приманку вместо того, чтобы атаковать реальную сеть, установленные на ней средства слежения и регистрации должны будут зафиксировать все подробности этого процесса:  выявить атаки, направить их по ложному следу, ограничить их распространение, идентифицировать нарушителей, исследовать их действия и определить намерения.

Цель данной работы — разработать систему автоматизации процесса создания инфраструктуры обманной системы в корпоративной сети.

Для достижения данной цели были поставлены следующие задачи: 

\begin{itemize}
	\item разработать архитектуру системы автоматического разворачивания ловушек;
	\item определить механизм взаимодействия между компонентами системы;
	\item разработать математическую модель сети с целью последующей выработки механизма определения количества ловушек в системе и их характеристик;
	\item разработать систему автоматического разворачивания ловушек.
\end{itemize}