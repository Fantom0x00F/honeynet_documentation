\section{Особенности применения обманных систем для обнаружения вторжений}

Обманная система (ОбС) представляют собой комплекс программно-аппаратных средств обеспечения информационной безопасности, реализующая функции сокрытия защищаемых информационных ресурсов, а также дезинформации нарушителей. С помощью фиксации, сбора данных, и обмана нарушителей (на основе имитации ложных целей, уязвимых для нападения), а также других механизмов эти системы позволяют в реальном времени выявлять атаки, направлять их по ложному следу, ограничивать их распространение, идентифицировать нарушителей, исследовать их действия и определять намерения\citep{Kotenko2014}.

Обманная система состоит из элементов, называемых ловушками, приманками или honeypot. Ловушка — это ресурс, задача которого подвергнуться атаке или несанкционированному исследованию. Помимо ловушек, обманная система должна включать в себя системы сбора и хранения данных и системы управления. 

Обманные системы предназначены для реализации следующих основных целей:
\begin{itemize}
\item ограничение атак на целевые (критически важные) системы за счет отвлечения нарушителя и “принятия огня на себя” (следствием чего является снижение эффективности атак, в том числе замедление их реализации, или их полное блокирование; это может позволить вовремя среагировать на распространение вирусов, сетевых червей и т.п.);
\item скрытное обнаружение (отслеживание) и исследование (оперативный анализ) атак и не авторизованной активности (издержки сокращаются за счет снижения числа ложных срабатываний, так как любой трафик, направленный на обманную систему вероятнее всего содержит действия нарушителя); 
\item мониторинг случаев несанкционированного доступа к системе и ее использования не по назначению;
\item реагирование на действия нарушителя с целью введения его в заблуждение \citep{Kotenko2014} \citep{Hernacki2004}.
\end{itemize}

Обманная система не должна участвовать в повседневной работе корпоративной сети, к ней не должно происходить каких-либо обращений от других узлов, за исключением подключений с целью администрирования. По данной причине обманная система имеет низкую вероятность ложных срабатываний, так как любое обращение будет расцениваться как неправомерное.

Обманная система не предпринимает активных действий по привлечению злоумышленника. Реагирование на действия нарушителя происходит только после его непосредственной атаки.

Одной из главных особенностей применения обманных систем является сложность выбора конфигурации \citep{rajan}. Настройка ОбС является нетривиальной задачей. Количество, расположение и конфигурацию ловушек необходимо выбирать так, чтобы их применение было максимально эффективным.

\subsection{Особенности реализации сетей приманок}

Под фактической реализацией сетей приманок понимается выбор технологий управления ложными целями в составе обособленного сегмента сети, а также конкретные алгоритмы функционирования всех подсистем ОбС.  Независимо от используемых при реализации ловушек технологий, обманная система в обязательном порядке должна содержать определенные  компоненты для полноценной работы.

\subsubsection{Система сбора и хранения данных}\hspace*{\fill} \\

Обманная система должна содержать систему сбора и хранения данных о деятельности злоумышленника в системе, в том числе о командах, инициированных им. Нарушители могут использовать шифрование, чтобы скрыть свои действия. Например, как только нарушитель проник на хост обманной системы, он может осуществлять удаленное администрирование системы с помощью SSH. Для решения этой проблемы можно использовать специальные модули ядра операционной системы, устанавливаемые на хостах, которые могут стать объектами атак. Эти модули накапливают информацию обо всей деятельности нарушителей. Информацию, которую собирают модули ядра, нельзя сохранять локально на хосте, поскольку возникает угроза обнаружения, удаления или модификации этой информации. Поэтому указанную информацию необходимо удаленно собирать на защищенной системе, причем так, чтобы нарушитель об этом не знал. От компонентов обманной системы требуется передавать собранную информацию в сеть. Однако злоумышленник может проанализировать трафик и обнаружить, что в пересылаемых пакетах содержатся сведения о его собственной деятельности. Чтобы воспрепятствовать этому, компонент должен маскировать пакеты, например, под трафик NetBIOS, передаваемый из других систем. Причем IP и MAC адреса отправителей и получателей могут маскироваться под адреса локального сервера Windows, а данные, содержащиеся в пакетах, — шифроваться. В этом случае даже если нарушитель осуществляет перехват и анализ пакетов, то для него они будут выглядеть как обычный трафик.

\subsubsection{Система восстановления и создания резервных копий}\hspace*{\fill} \\

Обманная система, как и любая другая структура в сфере информационных технологий, постоянно изменяется с течением времени. Добавления новых компонентов, изменения конфигурации существующих, поломки в результате критических ситуаций обязательно присутствуют в ее жизненном цикле.

Обнаружение новых уязвимостей в компонентах корпоративной сети также оставляет свой след и на элементах обманной системы. Данные моменты влекут за собой необходимость включения системы восстановления и создания резервных копий. 

Наличие резервных копий позволяет восстановить состояние ловушки в исходное после повреждения в результате атаки, безопасно производить ее модификацию. Благодаря им появляется возможность быстрого переноса обманной системы или дублирования ее компонентов.

Наличие резервной копии ловушки, полученной после проведения атаки, существенно упрощает процесс анализа, позволяет определить инструментарий злоумышленника, выявить его намерения и детально изучить следы его деятельности. Таким образом данная система позволяет создать образ поврежденного элемента для последующего его анализа в лабораторных условиях.

\subsubsection{Система удаленного управления состояния ловушки}\hspace*{\fill} \\

Изменения рабочих процессов, расширение штата сотрудников, процедуры увеличения производительности системы и многое другое приводят к необходимости модифицирования компьютерной сети. Любое добавление, удаление или изменение узла или хранящейся на нем информации может повлиять на оптимальность текущей конфигурации обманной системы и требует изменения расположения ловушек. В условиях больших масштабов предприятия данные модификации происходят достаточно часто, поэтому в случае ручного режима работы с обманной системой возникает потребность в человеческих ресурсах, затраты по которым практически сравнимы с добавлением нового сетевого узла. Таким образом, система удаленного управления, позволяющая производить добавление новых или удаления старых ловушек очевидно не является необходимой, но строго рекомендуемой. Система автоматизации позволит минимизировать риск того, что необходимые изменения будут отложены по вине человеческого фактора, а также существенно упростит жизнь системному администратору и специалисту по безопасности.

\subsection{Существующие реализации обманных систем}

В настоящее время существует несколько реализаций обманных систем. Наиболее известными являются\citep{honeynet}:
\begin{itemize}
\item Capture-HPC
\item Dionaea
\item Dockpot
\item Glastopf
\item HIHAT
\item HoneyC
\item Honeyd
\item Honeymole
\item Honeystick
\end{itemize}

Большинство перечисленных реализаций не получили своего распространения по причине достаточно сложного процесса настройки или большого количества требований к окружению. Остальные решения, не смотря на низкий порог вхождения, не предоставляют той гибкости, которая необходима для комфортной интеграции и использования. К примеру Honeyd достаточно прост в конфигурации и не требователен к своему окружению, предоставляет дистрибутивы как для UNIX, так и для Windows систем, однако его концепция изначально являлась весьма сомнительной: Honeyd позволяет эмулировать компьютерную сеть из нескольких ловушек, однако эмуляция производиться на одном хосте и нет возможности построения распределенной обманной системы.

Также ни один инструмент не предоставляет функционала подбора конфигурации обманной системы. Все реализации представляют собой конечный вариант ловушки, задача настройки которой остается на стороне пользователя.

Несомненно важным фактом является то, что ни одна из перечисленных систем не имеет поддержки со стороны разработчиков на данный момент, что сразу уменьшает степень доверия к инструментам со стороны потенциальных клиентов.

Компании, специализирующиеся на разработке систем защиты от киберугроз и анализе уязвимостей, также используют обманные системы в своей работе. К примеру "Лаборатория Касперского" оценивает степень ориентированности атак на устройства из систем \textit{IoT} при помощи установленных в глобальной сети Интернет ловушек, эмулирующих работу умных часов, \textit{smart TV}, роутеров, камер и других элементов Интернета Вещей \citep{kaspersky}. Однако данные разработки являются закрытыми коммерческими проектами и не представлены огласке.

Таким образом можно сделать вывод, что подходящего инструмента, позволяющего развернуть распределенную обманную систему в корпоративную сеть с малыми ручными затратами, и распространяющегося открытым либо коммерческим способом на данный момент не существует.