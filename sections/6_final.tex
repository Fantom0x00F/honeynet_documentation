\sectioncentered*{Заключение}
\phantomsection% исправляет нумерацию в документе и исправляет гиперссылки в pdf
\addcontentsline{toc}{section}{ЗАКЛЮЧЕНИЕ}

В ходе выполнения дипломной работы была разработана высокоуровневая архитектура системы автоматизации процесса создания инфраструктуры обманной системы, определены основные требования и необходимый функционал для ее полноценной работы.

Для элементов разрабатываемой системы был выбран протокол взаимодействия и определен механизм защиты от компрометации злоумышленником.

Для последующей автоматизации процесса определения количества и расположения была выработана математическая модель защищаемой компьютерной сети, произведена постановка оптимизационной задачи и предложен вариант ее решения.

В рамках дипломной работы по разработанной архитектуре была реализовано программное обеспечение, которое с использованием технологии контейнерной виртуализации позволяет автоматически создавать и изменять ловушки в защищаемой сети.

Таким образом, проведенные исследования помогли достичь  поставленной цели: разработать механизм автоматизации процесса разворачивания элементов обманной системы в корпоративной сети.

Исходя из полученных результатов можно сделать вывод, что с использованием технологии контейнерной виртуализации можно построить произвольную по масштабу и сложности обманную систему, а разработанные механизмы автоматизации позволяют достичь этого в достаточно короткие сроки.