\section{Архитектура системы автоматического разворачивания ОбС}

Прежде чем выделять компоненты системы и рассматривать варианты их взаимодействия необходимо перечислить основные требования, возлагаемые на разрабатываемую систему:
\begin{itemize}
\item ловушки должны будут функционировать на реальных хостах сети с целью исключения аппаратных затрат;
\item при изменениях топологии защищаемой сети необходимо без больших затрат по времени и ресурсам изменять количество и расположение ловушек. Следует учесть, что под топологией подразумевается не только логическая топология сети, но и особенности расположения в ней таких важных функциональных элементов, как: почтовые сервера, сервера баз данных и приложений, сервера печати и т.д.;
\item необходим процесс создания и восстановления резервных копий ловушек;
\item необходим мониторинг активности запущенных ловушек и удаленный съем информации в случае возникновения на них активности.
\end{itemize}

Из перечисленных требований следует факт необходимости централизованного управления и распределенного функционирования разрабатываемой системы. Таким образом, разрабатываемая система будет в обязательном порядке содержать два компонента:
\begin{itemize}
	\item компонент централизованного управления, далее называемый <<Центральный узел>>;
	\item компонент для обслуживания ловушки на конечном хосте, выполняющий операции изменения ее состояния по команде от первого компонента. Далее данный компонент будет обозначаться как <<агент>>.
\end{itemize}


\subsection{Выбор технологии для создания и разворачивания ловушек}

С учетом требования на запуск ловушек на реальных хостах сети, удобнее всего применять технологии виртуализации, которые позволяют использовать ресурсы одного физического сервера для создания нескольких виртуальных хостов.

В условиях большого количества ловушек актуальной является задача уменьшения потребления вычислительных ресурсов виртуальной машиной, поэтому целесообразно рассматривать технологии контейнерной виртуализации \citep{Joy2015}. При данной технологии ядро операционной системы поддерживает несколько изолированных экземпляров пространства пользователя, вместо одного. Эти экземпляры с точки зрения пользователя полностью идентичны реальному серверу. Для систем на базе UNIX, эта технология может рассматриваться как улучшенная реализация механизма chroot. Наиболее популярной системой контейнерной виртуализации является LXC \citep{lxc_doc}, однако использование ее в чистом виде несет за собой массу неудобств, поэтому следует рассматривать инструменты, позволяющие упростить работу с LXC контейнерами.

С недавнего времени технологии контейнерной виртуализации стали неотъемлемым инструментом в процессе разработки и поставки программного обеспечения, так как позволяют существенно упростить создание необходимого окружения для выполнения поставляемого приложения. Тот факт, что среда, в которой приложение разрабатывается, будет соответствовать среде эксплуатации вдохновило \textit{IT} индустрию на создание вспомогательных инструментов для автоматизации работы с виртуализацией на уровне операционной системы.

Несомненным фаворитом среди контейнерных систем является \textit{Docker}.
\textit{Docker} является программным обеспечением для автоматизации процессов разворачивания и управления приложением в среде виртуализации на уровне операционной системы, позволяет упаковывать приложение со всем необходимым окружением в контейнер и предоставляет программный интерфейс для его управления. Данное ПО прекрасно подходит на роль инструмента для создания ловушек, так как является активно поддерживаемым и предоставляет большое количество готовых контейнеров посредством облачной системы \textit{Docker Hub}.

Перечислим особенности данного программного обеспечения, которые покрывают некоторые требования к разрабатываемой системе \citep{Boettiger2015}:

\begin{itemize}
\item  контейнер Docker может быть перезапущен без сохранения своего состояния. Благодаря данному факту отпадает необходимость в восстановлении и создании резервных копий, так как сам по себе docker образ является резервной копией, а восстановление осуществляется  операцией пересоздания контейнера из образа;
\item  простой механизм задания конфигурации для Docker образа. Детали окружения запускаемых в контейнере приложений, а также непосредственно множество приложений с параметрами их запуска описываются в одном конфигурационном файле. Параметры сети и объемы выделяемых под контейнер ресурсов указываются в команде запуска контейнера;
\item  программный интерфейс Docker предоставляет команды для отслеживания и съема изменений в файловой системе контейнера.
\end{itemize}

С учетом перечисленных особенностей можно сделать вывод о том, что Docker является подходящим инструментом для разрабатываемой системы.

Для каждой ловушки необходимо сформировать соответствующий docker образ, который будет содержать в себе все необходимые для ее работы компоненты. В задачи системы автоматизации разворачивания будет входить назначение соответствующего образа на заданный хост в компьютерной сети и последующий запуск контейнера.

\subsection{Агенты}

В компьютерной сети могут находиться узлы, которые не обладают достаточными ресурсными возможностями для запуска на ней контейнера с ловушкой. Факт обладания такими возможностями не может быть обнаружен автоматически и задача выделения хостов является задачей системного администратора.

На доступных к разворачиванию контейнеров хостах системному администратору необходимо будет установить агента, в обязанности которого будут входить:

\begin{itemize}
\item копирование необходимого docker образа с указанного ресурса;
\item изменение состояния docker контейнера по команде от некоторого центрального узла;
\item мониторинг активности в запущенном контейнере и информирование центрального узла в случае ее возникновения;
\item передача информации о состоянии контейнера центральному узлу при получении соответствующих запросов.
\end{itemize}

Наличие агента на хосте не говорит о наличии на нем ловушки. Агент выступает посредником между центральным узлом и приманкой, сам по себе потребляет минимальное количество ресурсов и, в случае отключенной ловушки, никаким образом не влияет на производительность.

Совокупность всех агентов в сети определяет подмножество узлов, на которых обманная система может производить необходимые операции. В случае изменения назначения узлов достаточно удаления агента из системы автоматизации и с обслуживаемого хоста.

Единственным требованием выдвигаемым к хосту является наличие на нем установленного дистрибутива \textit{Docker}, необходимого для функционирования ловушки.

Таким образом алгоритм внедрения системы автоматического разворачивания состоит из следующих шагов:

\begin{itemize}
	\item системный администратор выделяет узлы компьютерной сети, ресурсные возможности которого позволяют запустить на них ловушки;
	\item выделяется хост под центральный узел;
	\item на каждый выделенный администратором узел устанавливаются агент и, по необходимости, \textit{Docker}, настраивается соединение с центральным узлом.
\end{itemize}


\subsection{Центральный узел}

В обязанности центрального узла системы, как управляющего агентами компонента, будут входить следующие пункты:

\begin{enumerate}
	\item регистрирование добавленных агентов в системе;
	\item мониторинг активности агента, в который входят следующие проверки:
		\begin{enumerate}
			\item проверка доступности агента;
			\item проверка работоспособности хоста в случае отсутствия связи с установленным на нем агентом. Случай, в котором с агентом отсутствует соединение, но соответствующий хост является доступным, считается критичным;
			\item получение состояния контейнера на агенте. Возможны три варианта состояния: отключен, включен и активность отсутствует, включен и активность присутствует.	
		\end{enumerate}
	\item управление состоянием контейнера на агенте. Предусматривается следующий набор команд:
		\begin{enumerate}
			\item задание другого docker образа на агента;
			\item запуск назначенного контейнера с указанными параметрами;
			\item отключение запущенного контейнера;
		\end{enumerate}
	\item прием от агента и последующее сохранение информации об активности в контейнере.
\end{enumerate}

Помимо перечисленного, центральный узел должен на основании топологии компьютерной сети и расположения доступных агентов предоставлять пользователю потенциально наиболее оптимальное расположение и конфигурации ловушек.